% sem númeração no glossário
% espaçamento simples
\chapter*{GLOSSÁRIO}\label{gloss}
\singlespacing
\noindent
Acionamentos a Velocidade Variável (AVVs) ou Adjustable Speed Drives - Although de motors are still used in many entries, they are being replaced by variable-frequency drives controlling squirrel-cage induction motors. The advantage of a motor compared to an ac motor is the fact that the speed of the dc motors can be controlled. Variable-frequency driver can give these same speed control and torque characteristics to squirrel-cafe induction motors \cite{SMI04}.

\noindent
\textbf{Aterramento -} É a ligação intencional da carcaça de um equipamento elétrico com a terra \cite{COR05}. 

\noindent
\textbf{Banco de capacitores -} Os capacitores podem ser ligados em várias configurações, formando bancos, cujo número de células deve ser limitado, em função de determinados critérios \cite{FIL05}.

\noindent
\textbf{Capacitor -} O capacitor é um componente de dois terminais usado para modelar um dispositivo construindo por duas placas condutoras separadas por um material isolante \cite{DOR08}.

\noindent
\textbf{Carga -} Quantidade de eletricidade responsável pelos fenômenos elétricos \cite{DOR08}.

\noindent
\textbf{Ciclo conversores -} É um tipo de conversor, empregado especialmente para acionar grandes motores síncronos e assíncronos a frequências relativamente baixas. é composto de um inversor que fornece uma tensão alternada á onda quadrada, a uma frequência 10 vezes superior aquela que se quer obter, e de 3 conversores monofásicos \cite{FIN02}.

\noindent
\textbf{Compensadores síncronos -} O compensador é uma máquina síncrona, em que não há nenhuma troca de energia entre o sistema elétrico e o sistema mecânico. Um conjunto mecânico gira livremente e apenas precisa de uma pequena potencia ativa para manter o atrito necessário para vencer a rotação e rolamento do ar \cite{ANT03}.

\noindent
\textbf{Componente de sequência negativa e positiva -} Fazem parte do método das componentes simétricas, que transformam um conjunto de correntes e tensões desequilibradas em componentes de sequência positiva, negativa e zero \cite{CAB10}.

\noindent
\textbf{Conexão delta -} Circuito trifásico no qual os enrolamentos do sistema são ligados em forma de um anel fechado e a voltagem instantânea nos pontos do anel é igual a zero. Este sistema é usado apenas para circuitos ou geradores trifásicos \cite{GAR82}.

\noindent
\textbf{Conexão delta-estrela -} Delta e estrela são dois tipos de conexões onde o arranjo dos componentes possui uma disposição especifica. Uma conexão delta-estrela é a união dessas conexões respectivamente \cite{SAD03}.

\noindent
\textbf{Conexão estrela -} Um circuito polifásico no qual todas as trilhas da corrente do circuito se estendem de um terminal de entrada para um terminal ou condutor comum, que pode ser o neutro \cite{GAR82}.

\noindent
\textbf{Corrente -} A taxa de variação do fluxo de carga elétrica em um ponto dado \cite{DOR08}.

\noindent
\textbf{Corrente continua -} É uma corrente de valor constante \cite{DOR08}.

\noindent
\textbf{Corrente de falta (Corrente de curto-circuito) -} Um curto-circuito pode ser definido como uma conexão intencional ou acidental, em geral de baixa impedância, entre dois ou mais pontos que normalmente estão em diferentes potenciais elétricos. Como consequência, resulta uma corrente elétrica que pode atingir valores muito elevados, dependendo do tipo de curto-circuito. A esta corrente dá-se o nome de corrente de curto-circuito ou corrente de falta \cite{SAN09}.

\noindent
\textbf{Curtos circuitos fase-terra -} O curto-circuito entre a fase A e terra (fase-terra ou monofásico) \cite{SAN09}.

\noindent
\textbf{Conteúdo espectral -} No sentido de Espectro eletromagnético possui a seguinte definição: No vácuo, as ondas eletromagnéticas se propagam com a mesma rapidez e diferem entre si nas suas frequências. A classificação das ondas eletromagnéticas, baseada na frequência, constitui o espectro eletromagnético \cite{HET02}.

\noindent
\textbf{Controladores lógico programáveis (CLPs) -} Um controlador programável é uma máquina de controle baseada na lógica digital de estado sólido e constituída de subsistemas de computador e planejada em primeiro lugar para substituir relés eletromecânicos em aplicações em que a religação é necessária, devida a mudanças periódicas de sequência \cite{GAR82}.

\noindent
\textbf{Conversor CC-CC -} O emprego combinado de inversores e retificadores permite efetuar conversores de corrente continua/corrente continua. Esta conversão de corrente continua em corrente continua se torna necessária quanto, de um alimentador à c.c a uma determinada tensão, se deseja obter um tensão diversa: nas redes a c.c, não é, de fato, possível o emprego de transformadores \cite{FIN02}.

\noindent
\textbf{Conversores estáticos de frequência -} Os conversores estáticos são sistemas que realizam a função de conversão da energia elétrica de uma forma a outra se valendo para isto da característica de comutação dos interruptores de potência. O controle desta transferência de energia é obtido ao serem aplicados sinais de controle nestes interruptores a fim de modificar os seus tempos de condução \cite{ROE02}.

\noindent
\textbf{Descargas atmosféricas -} Descarga elétrica de origem atmosférica entre uma nuvem e a terra ou entre nuvens, consistindo em um ou mais impulsos de vários quiloamperes \cite{NBR54}.

\noindent
\textbf{Disfunção de controle -} Falhas no controle dos CLPs \cite{GAR82}.

\noindent
\textbf{Dispositivos a arco -} Arco elétrico é a fonte de calor mais utilizada na soldagem por fusão de materiais metálicos, pois apresenta uma combinação ótima de características, incluindo uma concentração adequada de energia para a fusão localizada do metal de base, facilidade de controle, baixo custo relativo do equipamento e um nível aceitável de riscos á saúde dos seus operadores. Atualmente equipamentos desse tipo são de grande importância industrial e utilizada na fabricação dos mais variados componentes e estruturas metálicas, e também na recuperação de peças danificadas ou desgastadas \cite{MOD09}.

\noindent
\textbf{Dispositivos de eliminação de faltas -} Existem dispositivos capazes de detectar e disparar sinais para interromper a linha de transmissão em que houve esta falta. Estes dispositivos são conhecidos como equipamentos de proteção e são responsáveis pela detecção e eliminação de faltas ocorridas, e devem operar no menor tempo possível, evitando que a integridade física do sistema seja comprometida devido a estas faltas \cite{SOU08}.

\noindent
\textbf{Distúrbios geomagnéticos -} \textit{The geomagnetic disturbance occurs when the magnetic field embedded in the solar  wind  is opposite  that  of  the  earth.  This  disturbance,  which  results in distortions  to  the  earth’s magnetic field,  can be of varying  intensity and has in the past impacted the operation of pipelines, communications systems, and electric power  systems} \cite{BAR91}.

\noindent
\textbf{DSP -} Trata da representação de sinais em forma digital, do processamento desses sinais e da informação que eles transportam \cite{HAD99}.

\noindent
\textbf{Efeito Ficlker -} Efeito de tremulação, em um tubo a vácuo, pequenas variações na corrente anódica que se supõe serem provocadas pela emissão de íons positivos pelo cátodo \cite{GAR82}.

\noindent
\textbf{Eletrônica de potência -} \textit{Power electronics involves the study of eletronic circuits intended to control the flow of electrical energy. These circuits handle power flow at levels much higher than the individual device ratings }\cite{RAS11}. 

\noindent
\textbf{Falta -} Em um circuito elétrico é caracterizada por qualquer falha que interfira no fluxo de corrente deste circuito. \cite{GRA96}.

\noindent
\textbf{Falta (permanente ou temporária) -} As faltas temporárias são, na sua grande maioria, devido à ocorrência de descargas atmosféricas sobre o sistema elétrico, as quais não resultam em danos permanentes no sistema de isolação. O sistema pode ser restabelecido tão rápido quanto o tempo de eliminação de falta pelos equipamentos de proteção. Já as faltas permanentes são causadas por danos físicos em algum elemento de isolação do sistema. É necessária a intervenção da equipe de manutenção da rede elétrica \cite{CAR97}.

\noindent
\textbf{Faltas fase-terra -} Tipo de falta de curto-circuito que ocorre entre um das fases e a terra \cite{CAR97}.

\noindent
\textbf{Fonte -} É um gerador de tensão ou corrente capaz de fornecer energia a um circuito \cite{DOR08}.

\noindent
\textbf{Fontes chaveadas -} Uma fonte chaveada, switched mode power supply, é um conversor CC-CC com uma entrada de tensão não regulada e uma saída de tensão regulada \cite{FIN02}.

\noindent
\textbf{Frequência fundamental -} A menor frequência que um cristal pode vibrar eficazmente e produzir uma saída. Esta frequência é dependente do material da constante K do cristal e de sua espessura t onde. f = K/t \cite{MAL07}.

\noindent
\textbf{Fusíveis -} Um dispositivo fusível é um dispositivo de proteção que, pela fusão de uma parte especialmente projetada, abre o circuito no qual se acha inserido e interrompe a corrente, quando esta excede um valor de referência durante um tempo especificado \cite{COR05}.

\noindent
\textbf{Impedância -} \textit{The impedance Z of a circuit is the ratio of the phasor voltage V to the phasor current I, measured in ohms} \cite{SAD03}.

\noindent
\textbf{Interrupção sustentada -} Considera-se como sendo uma interrupção sustentada o decréscimo a zero do valor RMS da tensão durante um intervalo de tempo que ultrapasse 1 minuto \cite{JUN09}.

\noindent
\textbf{Lógica Fuzzy -} A lógica \textit{fuzzy} é a lógica baseada na teoria dos conjuntos fuzzy. Ela difere dos sistemas lógicos tradicionais em suas características e seus detalhes. Nesta lógica, o raciocínio exato corresponde a um caso limite do raciocínio aproximado, sendo  interpretado como um processo de composição de relações  nebulosas.  Na lógica \textit{fuzzy}, o valor verdade de uma proposição pode ser um subconjunto \textit{fuzzy} de qualquer conjunto parcialmente ordenado, ao contrário dos sistemas lógicos binários, onde o valor verdade só pode assumir dois valores: verdadeiro (1) ou falso (0) \cite{GOM94}. 

\noindent
\textbf{Motores de indução -} Os motores de indução funcionam velocidade praticamente constante abaixo da velocidade síncrona, e variam ligeiramente com a carga mecânica aplicada ao eixo. Devido à sua robustez e ao baixo custo, são os motores mais utilizados, principalmente os tipos gaiola e são adequados para a maioria dos encontrados da indústria \cite{COR05}. 

\noindent
\textbf{Potência -} Taxa com qual a energia é fornecida ou absorvida \cite{DOR08}.

\noindent
\textbf{Queda de tensão -}\textit{The definition of voltage drop is the voltage difference between any two points of circuit or conductor, due to the flow of current} \cite{LOC08}.

\noindent
\textbf{RNA -} A expressão \textit{rede neural} é motivada pela tentativa destes modelos imitarem a capacidade que o cérebro humano possui de reconhecer, associar e generalizar padrões. Trata-se de uma importante técnica estatística não-linear capaz de resolver uma gama de problemas complexos. Isso torna o método extremamente útil quando não é possível definir um modelo explícito ou uma lista de regras. Em geral, isso acontece em situações em que o ambiente dos dados muda muito. As principais áreas de atuação são para a classificação de padrões e previsão \cite{VEL07}.

\noindent
\textbf{Regime permanente -} É o intervalo de tempo da leitura de tensão, onde não ocorrem distúrbios elétricos capazes de invalidar a leitura, definido como sendo de 10 (dez) minutos \cite{ANE01}.

\noindent
\textbf{Religadores automáticos -} São equipamentos de interrupção da corrente elétrica dotados de uma determinada capacidade de repetição em operações de abertura e fechamento de um circuito, durante a ocorrência de um defeito \cite{FIL05}. 

\noindent
\textbf{Resistência -} Propriedade física de um componente ou dispositivo que se opõe à passagem de corrente elétrica, é representada pelo símbolo R \cite{DOR08}.

\noindent
\textbf{Retificadores -} São circuitos pertencentes a uma fonte de alimentação que permite que a corrente circule em apenas um sentido. Esses circuitos convertem a forma de onda CA na entrada em uma forma de onda pulsante CC na saída \cite{MAL07}.

\noindent
\textbf{Retificação de meia onda -} Um circuito que contem um retificador com apenas um diodo em série com um resistor de carga. A saída é uma tensão retificada em meia onda \cite{MAL07}.

\noindent
\textbf{Sistema Computacional (\textit{Computer System}) -} \textit{A System that (a) consists of one or more computers and associated software, (b) uses common storage for (i) all or part of a computer program and (ii) all or part of the data necessary for the execution of the program, (c) executes user-written or user-designated programs, (d) performs user-designated data manipulation, including arithmetic and logical operations, (e) may execute programs that modify themselves during their execution, and (f) my be a standalone system or may consist several interconnected systems. Synonym Computing System} \cite{WEI00}.

\noindent
\textbf{Sistema \textit{Neurofuzzy} -} O termo \textit{neurofuzzy} refere-se à combinação de RNAs e \textit{Fuzzy Inference System}, resultando em um sistema inteligente híbrido que potencializa as características destes dois importantes paradigmas \cite{JAC10}.

\noindent
\textbf{Sobtensões -} define-se como sendo sobretensões sustentadas o acréscimo da ordem de 1,1 a 1,2 pu da tensão eficaz, à frequência da rede, sustentada por um intervalo de tempo superior a 1 minuto \cite{JUN09}.

\noindent
\textbf{Subtensões -} Podem-se definir as subtensões sustentadas como a redução para valores entre 0,8 a 0,9 pu da tensão eficaz, à frequência da rede, sustentada por um intervalo de tempo superior a 1 minuto \cite{JUN09}.

\noindent
\textbf{Tensão -} A tensão ente os terminais de um componente é o trabalho (energia) necessário (a) para transformar uma unidade de carga positiva no terminal - para o terminal +. A unidade de tensão é o volt \cite{DOR08}.

\noindent
\textbf{Tensões Trifásicas -} São compostas de três fases, essas tensões possuem amplitude é igual e estão fora de fase em relação aos outras em 120º \cite{SAD03}.

\noindent
\textbf{Torque -} É definido como a tendência do acoplamento mecânico (de uma força e sua distância radial ao eixo de rotação) para produzir rotação \cite{KOS93}. 

\noindent
\textbf{Transformadores -} O transformador opera segundo o princípio da indução mútua entre duas (ou mais) bobinas ou circuitos indutivamente acoplados, ou seja, é um dispositivo estático que transporta energia elétrica, por indução eletromagnética, da bobina primária (entrada) para a secundária (saída) \cite{KOS93}. 

\noindent
\textbf{Transformadores de tap variável -} Muitos transformadores utilizados em sistemas elétricos de potência possuem uma tomada, denominada tap. Um transformador com tap variável tem a relação entre as espiras variável em relação ao valor nominal para compensar variações das tensões no sistema \cite{FER11}.

\noindent
\textbf{Valor eficaz -} É também denominado valor quadrático médio ou RMS é uma medida estatística da magnitude de uma quantidade variável de dados para um intervalo igual ao período \textit{T} \cite{BOY04}.