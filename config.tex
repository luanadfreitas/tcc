\usepackage[english,brazilian]{babel}
% Codificação do arquivo
\usepackage[utf8]{inputenc}
% Mapear caracteres especiais no PDF
\usepackage{cmap}
% Codificação da fonte
\usepackage[T1]{fontenc}
% Essencial para colocar funções e outros símbolos matemáticos
\usepackage{amsmath,amssymb,amsfonts,textcomp}
\usepackage[left=3cm,right=2cm,top=3cm,bottom=3cm]{geometry}
\usepackage{setspace}
\usepackage{times}
\usepackage{fancyhdr}
\setcounter{secnumdepth}{2}

% para gerar lista de abreveaturas 
% link http://comments.gmane.org/gmane.comp.tex.brazilian/2162
\usepackage{acronym}

\usepackage{color}
% usar links coloridos e outras links no doc
\usepackage[pdftex]{hyperref}
% deixar link sem cores
\hypersetup{colorlinks,
debug=false,
linkcolor=black, %%% cor do tableofcontents,
citecolor=black, %%% cor do \cite
urlcolor=black, %%% cor do \url e \href
pdftoolbar=true,
bookmarks=true,
unicode=true,
pdftitle={TC2 - Freitas},
pdfauthor={Luana Lima de Freitas},
pdfsubject={Sistema Computacional para Tomada de Decisões e Gerenciamento da Energia Elétrica em Sistemas Industriais},
pdfkeywords={EE,QEE,Sistemas Embarcado, Tomada de decisão}
}

%% Lista de Abreviações
% Cria lista de abreviações
\usepackage[notintoc,portuguese]{nomencl}
\makenomenclature

%############################################################################
%% Referências bibliográficas e afins
% Formatar as citações no texto e a lista de referências
\usepackage{natbib}

% Adicionar bibliografia, índice e conteúdo na Tabela de conteúdo
% Não inclui lista de tabelas e figuras no índice
\usepackage[nottoc,notlof,notlot]{tocbibind}

%% Pontuação e unidades
% Posicionar inteligentemente a vírgula como separador decimal
\usepackage{icomma}

% Formatar as unidades com as distâncias corretas
\usepackage[tight]{units}

%
%############################################################################
%% Comandos customizados
% Espécie e abreviação
\newcommand{\subde}{\emph{Clypeaster subdepressus}}
\newcommand{\subsus}{\emph{C.~subdepressus}}
% Dados do Trabalho
\newcommand{\nomedocurso}{curso de engenharia de computação}
\newcommand{\nomedoaluno}{Luana Lima de Freitas}
% Palavras Chaves do Trabalho
\newcommand{\titulo}{sistema computacional para tomada de decisões e gerenciamento da energia elétrica em sistemas industriais}
\newcommand{\palavras}{Qualidade da Energia Elétrica, Eficiencia Energetica, Sistemas Embarcado, Tomada de decisão.}
\newcommand{\wordskey}{Quality Electrical Energy, Energy Efficiency, Embedded System, Decision Making.}
\newcommand{\capum}{Introdução}
\newcommand{\capdois}{Qualidade da Energia Elétrica}
\newcommand{\captres}{Análise de Sinais}
\newcommand{\capquatro}{Estruturas Computacionais}
\newcommand{\capcinco}{Metodologia}
\newcommand{\capsex}{Resultados}
%############################################################################
%% Pacotes não implementados
% Para não sobrar espaços em branco estranhos
\widowpenalty=1000
\clubpenalty=1000
%############################################################################
