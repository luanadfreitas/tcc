% Limpa cabeçalhos.
% (solução para lidar com a númeração das páginas pré-textuais).
\pagestyle{empty}
% Faz com que a página seguinte sempre seja ímpar (insere pg em branco)
\cleardoublepage
% Numeração em elementos pré-textuais é opcional (ativada por padrão).
% Para desativá-la comente a linha abaixo.
%\pagestyle{fancy}
% Números das páginas em algarismos arabic
%\pagenumbering{arabic}

% Dedicatória
% Posição do texto na página
\vspace*{0.75\textheight}
\begin{flushright}
  \emph{Dedicatória...}
\end{flushright}
\newpage

% Agradecimentos
% Espaçamento duplo
\doublespacing
\noindent{\LARGE\textbf{Agradecimentos}}
\par
Agradeço ao meu orientador, ao meu co-orientador, aos meus colaboradores, aos técnicos, à seção administrativa, à fundação que liberou verba para minhas pesquisas, aos meus amigos, à minha família e ao meu grande amor.
\newpage
\vspace*{10pt}

%#############################################################################
% Resumo e Abstract
%#############################################################################
\begin{center}
\emph{\begin{large}Resumo\end{large}}\label{resumo}
\vspace{2pt}
\end{center}
% Pode parecer estranho, mas colocar uma frase por linha ajuda a organizar e reescrever o texto quando necessário.
% Além disso, ajuda se você estiver comparando versões diferentes do mesmo texto.
% Para separar parágrafos utilize uma linha em branco.
\par
O constante crescimento econômico pelo qual passa o país tem influenciado diretamente no aquecimento do setor industrial. 
Isto se reflete na modernização de parques e processos industriais, os quais trazem consigo uma demanda considerável de energia elétrica.
A gestão de energia elétrica é um problema que atinge a sociedade em geral e diversos segmentos da atividade humana.
Tal fato influi diretamente em despesas para a sua manutenção, assim como no impacto ambiental de uma região.
A Qualidade de Energia Elétrica (QEE) e a Eficiência Energética (EE) são dois importantes parâmetros a serem considerados quando se trata de gestão energética e o desenvolvimento de sistemas computacionais que auxiliem na análise, monitoração e controle desses parâmetros.
Neste contexto, este trabalho tem como principal objetivo a implementação de um método de tomada de decisão e gerenciamento da energia elétrica como ferramenta de auxílio em processos industriais.
\par
\vspace{1cm}
\noindent\textbf{Palavras-chave:} {\palavras}
\newpage

% Criei a página do abstract na mão, por isso tem bem mais comandos do que o resumo acima, apesar de serem idênticas.
\vspace*{10pt}
% Abstract
\begin{center}
\emph{\begin{large}Abstract\end{large}}\label{abstract}
\vspace{2pt}
\end{center}
\selectlanguage{english}
% Selecionar a linguagem acerta os padrões de hifenação diferentes entre inglês e português.
\par
The constant economic growth which passes through the country has directly influenced the heating industry. This is reflected in the upgrading of parks and industrial processes, which bring considerable demand for electricity. The management of electric power is a problem that affects society in general and various segments of human activity.
This fact has a direct influence on costs for maintenance, as well as the environmental impact of a region.
The Quality Electrical Energy (QEE) and Energy Efficiency (EE) are two important
parameters to consider when it comes to energy management and the development of
computer systems that assist in the analysis, monitoring, and control of these parameters. 
In this context, this paper has as main objective the implementation of a method of decision making and management of electrical energy as a tool to aid in industrial processes.
\par
\vspace{1cm}
\noindent\textbf{Keywords:} {\wordskey}
%#############################################################################
% Voltando ao português...
\selectlanguage{brazilian}
\newpage


% Lista de figuras
\listoffigures

% Lista de tabelas
\listoftables

% Abreviações
% Para imprimir as abreviações siga as instruções em 
% http://code.google.com/p/mestre-em-latex/wiki/ListaDeAbreviaturas
\printnomenclature

% Índice
\tableofcontents
