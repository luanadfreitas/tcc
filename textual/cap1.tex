% Capitulo 1 - Introdução
\cleardoublepage
\pagestyle{fancy}
%O comando \setcounter, associado ao que queremos alterar, permite essa alteração de uma forma muito simples. Introduzimos no local onde queremos alterar a página o seguinte comando: \setcounter{page}{3}. O que significa que pomos o contador da página no número 3.
\setcounter{page}{9}
\chapter{\capum}\label{intro}
\par
Mobilizar a sociedade para o uso eficiente da energia elétrica, combatendo o seu desperdício é a missão estratégica do \ac{PROCEL}. A economia de energia elétrica proporciona inúmeras vantagens, entre elas a liberação de recursos para outras áreas e a contribuição para a preservação do meio ambiente \cite{ELE10}.
\par
Os fatores que influenciam na \ac{QEE} podem ser originados tanto nas concessionárias como nos sistemas consumidores \cite{MEL08}. Estes distúrbios podem ser gerados por fenômenos naturais, por operações da concessionária ou pelos próprios consumidores. Da mesma forma que uma maior demanda em horário de ponta causa perturbações e desequilíbrio na rede, além de penalizações por parte da \ac{ANEEL} caso o consumo ultrapasse o limite contratado. 
\par
Contextualizando a \acl{QEE} no cenário industrial, torna-se importante que as indústrias tenham um gerenciamento eficaz das demandas energéticas consumidas diariamente, evitando tanto o desperdício de energia quanto a diminuição da qualidade da mesma. Tendo em vista que a QEE pode chegar ao consumidor com um excelente padrão de qualidade, o mau gerenciamento, ou até mesmo problemas decorrentes da própria infraestrutura da empresa, podem implicar na diminuição da qualidade da energia consumida no processo industrial. 
\par
Conforme \cite{SIL09} e \cite{SOL04}, as indústrias são peças importantes no contexto estudado, devido ao grande consumo de energia elétrica necessário em seus processos de produção. Em função disto, \cite{SOL04} define como pontos relevantes para a pesquisa dentro desta área:
%% Itens em topicos
\begin{itemize}
	\item Conscientização para \ac{EE};
	\item Gestão energética eficaz;
	\item Política de uso de tecnologias energeticamente eficientes;
	\item Monitoração da qualidade da energia consumida.
\end{itemize}
\par
Com isso, justifica-se o estudo e desenvolvimento de um sistema que implementa uma metodologia de gerenciamento de energia elétrica que atenda aos requisitos propostos, servindo de apoio à tomada de decisão e controle, além de uma solução eficiente em termos de \ac{EE} para o setor industrial \cite{GAR09}.
\par
Este trabalho além de focar no desenvolvimento de um sistema que proporcione o gerenciamento da energia elétrica, também oferece a tomada de decisão não sendo somente uma ferramenta de apoio, mas sim um controle inerente de operadores e avaliadores. A linha de simulação desde trabalho deve proporcionar aos usuários a realização de perguntas \emph{when-if}, buscando apontar outros comportamentos do sistema elétrico dadas situações especificas.
\par
O objetivo principal deste trabalho é o desenvolvimento de um sistema de tomada de decisão e gerenciamento da energia elétrica, aplicado a sistemas industriais. Os objetivos específicos para direcionamento do trabalho proposto seguem listados abaixo:
%% Itens em topicos
\begin{itemize}
	\item Realizar estudo aprofundado sobre parâmetros QEE, distúrbios eletromagnéticos e dos métodos utilizados na identificação dos mesmos;
	\item Realizar a modelagem matemática e simulação do sistema;
	\item Estudar e implementar  um modelo de comunicação para o sistema;
	\item Desenvolver os algoritmos ligados à tomada de decisão e \textit{software} de gerenciamento de \ac{QEE};
	\item Implementar a tomada de decisão e \textit{software} de gerenciamento no sistema.
\end{itemize}
\par
\todo[color=red]{Descrever novamente a introdução aos contéudos dos seguintes capítulos}%
O presente trabalho encontra-se dividido em três capítulos, o capítulo 2 contém o referencial teórico desde trabalho e aborda a \acl{QEE} em sua conceituação, normalização e índices, juntamente com os distúrbios eletromagnéticos causadores e motivadores deste controle de qualidade. 
Em seguida, neste mesmo capítulo, trata-se da análise de sinais, considerando três ferramentas de trabalho, e são abordados os conceitos gerais que serão a base computacional para construção do sistema proposto. No capitulo 3 consta a explanação da metodologia proposta e seu respectivo cronograma.